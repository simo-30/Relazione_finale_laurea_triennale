\documentclass[Lau, oneside]{sapthesis}%remove "english" for a thesis written in Italian
%Bachelor's (laurea triennale) thesis : Lau 
%Master's (laurea specialistica) thesis: LaM 
%PhD's thesis: PhD 
\usepackage[italian]{babel} %use this package for a thesis written in Italian
\usepackage[utf8]{inputenx}
\usepackage{indentfirst}
\usepackage{microtype}
%\usepackage{chemformula}
%\usepackage{setspace}
%\usepackage{yfonts,color}
%\usepackage{siunitx}
%\usepackage{comment}
%\usepackage{multirow}
%\usepackage{varioref}
%\usepackage[bottom]{footmisc}
%\usepackage{wrapfig}
%\usepackage{float}
%\usepackage{type1cm}
\usepackage{lettrine}
\linespread{0.9}
%\usepackage{chngcntr}
\usepackage[nottoc, notlof, notlot]{tocbibind}
%\onehalfspacing
%\counterwithout{footnote}{chapter}
\usepackage{hyperref}
\hypersetup{
			hyperfootnotes=true,			
			bookmarks=true,			
			colorlinks=true,
			linkcolor=red,
                        linktoc=page,
			anchorcolor=black,
			citecolor=red,
			urlcolor=blue,
			pdftitle={A sample Bachelor's thesis for Sapienza Università di Roma},
			pdfauthor={FirstName LastName},
			pdfkeywords={thesis, sapienza, roma, university}
}

\title{Scheduler dei processi di un sistema operativo, confronto ed analisi di varie politiche}
\author{Simone Trenta}
\IDnumber{1724141}
\course[]{Ingegneria informatica ed automatica}
\courseorganizer{Facolt\`a di Ingegneria dell'informazione, informatica e statistica}
\submitdate{2020/2021}
\copyyear{2020}
\advisor{Prof. Giorgio Grisetti}
\authoremail{trenta.1724141@studenti.uniroma1.it}
\examdate{22 September 2015}
\examiner{Prof. ...} \examiner{Prof. ...} \examiner{Prof. ...}  \examiner{Prof. ...}  \examiner{Prof. ...} \examiner{Prof. ...}  \examiner{Prof. ...} 

%we refer to http://ctan.mirrorcatalogs.com/macros/latex/contrib/sapthesis/sapthesis-doc.pdf for an exhaustive description of the sapthesis documentclass.


\begin{document}

\frontmatter
\maketitle

\begin{abstract}
Tramite un progetto di simulazione di uno scheduler di un sistema operativo, si analizzano le varie politiche di scheduling fra le più comuni, ossia first came first served (FCFS), round robin (RR), shortest job first (SJF) e shortest remaining job first (SRJF).
\end{abstract}

\tableofcontents

\mainmatter
\chapter{Introduzione}
\lettrine[lines=2, findent=3pt, nindent=0pt]{I}{}n un sistema operativo la gestione delle richieste di risorse da parte dei processi è demandata allo scheduler.
Questa funzione di un sistema operativo è pressoché essenziale, senza la quale le risorse disponibili non potrebbero essere riservate e utilizzate in modo efficiente.

\bigskip
Nel \hyperref[chap:1]{Capitolo~\ref*{chap:1}}

\bigskip
Nel \hyperref[chap:2]{Capitolo~\ref*{chap:2}} 

\bigskip
Nel \hyperref[chap:3]{Capitolo~\ref{chap:3}} 

\chapter{Lo scheduler in un sistema operativo}
\label{chap:1} 
\section{La sua utilità}
\label{sec:utilita}
Un odierno sistema operativo è composto da molteplici parti, una di esse è lo scheduler dei processi.
Questi ultimi sono parte costituente del sistema operativo, sono la parte esecutiva di esso.
Per meglio capire, i processi richiedono delle risorse del calcolatore, che siano risorse unicamente della CPU oppure di risorse di I/O.
Qualunque tipo di risorsa richiedano, spesso non è immediatamente disponibile, essendo limitate in un calcolatore.
Per cui è necessaria la presenza di un elemento che permetta al dispositivo di gestire le proprie risorse, tale elemento è per l'appunto lo scheduler.
Vedendo il problema da parte dei processi, problema che è la possibilità di usare una risorsa o meno, si crea la necessità di dare delle priorità per poterne usufruire.
Andando quindi a dare origine a tali priorità si producono di conseguenza delle politiche per poterle mettere in atto.

\section{prova}
\label{sec:prova}
sezione di prova

\chapter{MOONS}
\label{chap:2}
\section{The Very Large Telescope}
Property of the European Southern Observatory...

\section{The Multi-Object Optical and Near-infrared Spectrograph}
\label{sec:moons}

The \textit{Multi-Object Optical and Near-infrared Spectrograph} is a future generation MOS instrument for the VLT. 

\chapter{Conclusions}
\label{chap:3}
The grasping power of the mirror..

\backmatter
\phantomsection
\begin{thebibliography}{17}

\bibitem{ref:vph}
Blanche P.A., Gailly P., et al., “\textit{Volume phase holographic gratings: large size and high diffraction efficiency}“, Optical Engineering, Vol. 43, No.11, November 2004

\bibitem{ref:science}
Cirasuolo M., et al., \textit{"MOONS Science Report"}, MOONS Document Number: VLT-TRE-MON-14620-0001, Issue: $1.0$, $31^{\textup{st}}$ January $2013$

\bibitem{ref:eso}
European Southern Observatory, \url{http://www.eso.org}

\end{thebibliography}

\end{document}