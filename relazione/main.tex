\documentclass[Lau, oneside]{sapthesis}%remove "english" for a thesis written in Italian
%Bachelor's (laurea triennale) thesis : Lau 
%Master's (laurea specialistica) thesis: LaM 
%PhD's thesis: PhD 
\usepackage[italian]{babel} %use this package for a thesis written in Italian
\usepackage[utf8]{inputenx}
\usepackage{indentfirst}
\usepackage{microtype}
%\usepackage{chemformula}
%\usepackage{setspace}
%\usepackage{yfonts,color}
%\usepackage{siunitx}
%\usepackage{comment}
%\usepackage{multirow}
%\usepackage{varioref}
%\usepackage[bottom]{footmisc}
%\usepackage{wrapfig}
%\usepackage{float}
%\usepackage{type1cm}
\usepackage{lettrine}
\linespread{0.9}
%\usepackage{chngcntr}
\usepackage[nottoc, notlof, notlot]{tocbibind}
%\onehalfspacing
%\counterwithout{footnote}{chapter}
\usepackage{hyperref}
\hypersetup{
			hyperfootnotes=true,			
			bookmarks=true,			
			colorlinks=true,
			linkcolor=red,
                        linktoc=page,
			anchorcolor=black,
			citecolor=red,
			urlcolor=blue,
			pdftitle={A sample Bachelor's thesis for Sapienza Università di Roma},
			pdfauthor={FirstName LastName},
			pdfkeywords={thesis, sapienza, roma, university}
 }

\title{Scheduler dei processi di un sistema operativo, varie politiche, FCFS, RR, SJF, SRJF, MLFQ}
\author{Simone Trenta}
\IDnumber{1724141}
\course[]{Ingegneria informatica ed automatica}
\courseorganizer{Facolt\`a di Ingegneria dell'informazione, informatica e statistica}
\submitdate{2020/2021}
\copyyear{2020}
\advisor{Prof. Giorgio Grisetti}
\authoremail{trenta.1724141@studenti.uniroma1.it}
\examdate{22 September 2015}
\examiner{Prof. ...} \examiner{Prof. ...} \examiner{Prof. ...}  \examiner{Prof. ...}  \examiner{Prof. ...} \examiner{Prof. ...}  \examiner{Prof. ...} 

%we refer to http://ctan.mirrorcatalogs.com/macros/latex/contrib/sapthesis/sapthesis-doc.pdf for an exhaustive description of the sapthesis documentclass.


\begin{document}

\frontmatter
\maketitle

\begin{abstract}
Tramite un progetto di simulazione di uno scheduler di un sistema operativo, si analizzano le varie politiche di scheduling fra le più comuni.
Fra queste troviamo first came first served (FCFS), round robin (RR), shortest job first (SJF), shortest remaining job first (SRJF), multilevel feedback queue (MLFQ).
I risultati ottenuti tramite la simulazione verranno poi analizzati e confrontati, in particolare si metterà in risalto come la differenza fra le politiche di scheduling, determina vari tempi di attesa, nonostante gli stessi dati di input.
Tramite i sopra citati confronti si potrà poi determinare quale politica sia meglio per ogni calcolatore si voglia progettare, sapendo il lavoro che esso dovrà compiere.
\end{abstract}

\tableofcontents

\mainmatter
\chapter{Introduzione}
\lettrine[lines=2, findent=3pt, nindent=0pt]{I}{}n the frame of astronomical spectroscopy...

\bigskip
In \hyperref[chap:1]{Chapter~\ref*{chap:1}} we  briefly present...

\bigskip
In \hyperref[chap:2]{Chapter~\ref*{chap:2}} we summarize...

\chapter{Near-infrared multi-object spectroscopy}
\label{chap:1} 
\section{Scientific Case}
\label{sec:caso}
Over the last decades, innovative observational techniques have been developed to allow spectrographs observing...

\chapter{MOONS}
\label{chap:2}
\section{The Very Large Telescope}
Property of the European Southern Observatory...

\section{The Multi-Object Optical and Near-infrared Spectrograph}
\label{sec:moons}

The \textit{Multi-Object Optical and Near-infrared Spectrograph} is a future generation MOS instrument for the VLT. 

\chapter{Conclusions}
The grasping power of the mirror..

\backmatter
\phantomsection
\begin{thebibliography}{17}

\bibitem{ref:vph}
Blanche P.A., Gailly P., et al., “\textit{Volume phase holographic gratings: large size and high diffraction efficiency}“, Optical Engineering, Vol. 43, No.11, November 2004

\bibitem{ref:science}
Cirasuolo M., et al., \textit{"MOONS Science Report"}, MOONS Document Number: VLT-TRE-MON-14620-0001, Issue: $1.0$, $31^{\textup{st}}$ January $2013$

\bibitem{ref:eso}
European Southern Observatory, \url{http://www.eso.org}

\end{thebibliography}

\end{document}